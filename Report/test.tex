\documentclass[a4paper,12pt]{article}
\usepackage{ctex} % 中文支持
\usepackage{pdfpages} % 插入封面
\usepackage{geometry} % 页边距设置
\usepackage{titlesec} % 标题格式设置
\usepackage{titletoc} % 目录格式设置
\usepackage{fancyhdr} % 页眉页脚设置
\usepackage{abstract} % 摘要格式设置
\usepackage{amsmath} % 数学公式
\usepackage{amssymb} % 数学符号
\usepackage{booktabs} % 三线表
\usepackage{multirow} % 多行合并
\usepackage[style=gb7714-2015]{biblatex} % 参考文献格式设置
\usepackage{caption} % 图表标题字体
\usepackage{float} % 图表位置设置

% 页边距设置
\geometry{left=2.5cm,right=2.5cm,top=3cm,bottom=2.5cm}

% 标题格式设置
\titleformat{\section}{\Large\bfseries}{\thesection}{1em}{}
\titleformat{\subsection}{\large\bfseries}{\thesubsection}{1em}{}
\titleformat{\subsubsection}{\normalsize\bfseries}{\thesubsubsection}{1em}{}

% 目录格式设置
\dottedcontents{section}[0em]{\vspace{0.5em}}{1.5em}{0.5pc}
\dottedcontents{subsection}[2em]{}{2.3em}{0.5pc}
\dottedcontents{subsubsection}[4.4em]{}{3.2em}{0.5pc}

% 页眉页脚设置
\pagestyle{fancy}
\fancyhf{}
\fancyfoot[R]{\thepage}
\renewcommand{\headrulewidth}{0pt} % no upper line

% 摘要格式设置
\renewcommand{\abstractnamefont}{\large\bfseries} % 摘要标题字体
\renewcommand{\abstracttextfont}{\normalsize} % 摘要正文字体
\setlength{\absleftindent}{0pt} % 摘要左缩进
\setlength{\absrightindent}{0pt} % 摘要右缩进

% 正文行距设置
\linespread{1.44}

% 参考文献格式设置
\addbibresource{ref.bib}

% 开始文档
\begin{document}
\captionsetup{font={small}} % 图表标题字体
\includepdf[pages={1}]{src/cover.pdf}
% 生成标题页


\newpage
% 生成摘要页
\begin{abstract}
本文设计并制作了由2个单相逆变器组成的并联系统,
系统可以向电阻负载$R_L$供电,
也可以通过变压器T并入220V电网。
系统采用GD32F470ZGT6作为主控MCU,
使用H桥SPWM方式实现逆变原理,通过使用霍尔传感器进行电流采样,
过零比较器进行相位调整,使用PID算法进行电流电压控制,
从而监测与控制电路,使其满足并网要求。系统的性能经过测试和分析,
满足设计要求,具有较高的效率和稳定性,
能够实现逆变器的并联运行和并网运行。本设计各个模块布局合理,
系统稳定性好,制作成本低,经测试,
能够完成题目的基本要求与发挥部分。


\noindent \textbf{关键词:} 
SPWM 波逆变、增量式PID算法、过零比较器、并网、LC滤波
\end{abstract}

% 新起一页
\newpage

% 生成目录页
\tableofcontents

% 新起一页
\newpage

% 开始正文
\section{设计任务与要求}
设计并制作由2个单相逆变器组成的并联系统,
系统框图如图\ref{fig1}所示,
逆变器并联后可为电阻负载$R_L$供电,也可通过变压器T并入220V电网。
\begin{figure}[htbp]
\centering
\includegraphics[width=0.7\textwidth]{src/fig1.png}
\caption{单相逆变器并联示意图}
\label{fig1}
\end{figure}
\begin{enumerate}
    \item 断开S2,闭合S1,
    仅用逆变器1向$R_L$供电。$U_o=24V\pm0.2V\  f_o=50Hz\pm0.2Hz$时,$I_o=2A$。
    输出交流电压$THD\leqslant 2\%$。$\eta\geqslant 88\%$。
    $I_o$在$0A\sim 2A$间变化时,$S_{I1}\leqslant 0.2\%$。
    逆变器1和逆变器2并联时,
    共同向$R_L$供电,$U_o=24V$, $f_o=50Hz$ 时,$I_o=4A$。
    \item 断开S1,闭合S2,逆变器1与逆变器2并联且能并网,
    能在$2A\sim4A$范围内按数字设定$I_o$,
    其误差绝对值应小于设定值的$6\%$。
    $I_o$在$1A\sim 3A$间变化时,
    逆变器1及逆变器2的输出电流比值$K=I_{o1}:I_{o2}$
    可在$0.5\sim 2$内按数字设定自动分配,
    其相对误差的绝对值不大于$5\%$。
\end{enumerate}

\section{系统方案设计}
\subsection{设计方案的比较与选择}

\subsubsection{H桥设计方案}
\begin{enumerate}
    \item 使用EG2104S栅极驱动芯片和BSC070N10NS
    \\ 
    EG2104S是一种自举升压的栅极驱动芯片,有一路驱动信号的输入端
    有较强的拉灌电流能力,有一个驱动信号的输入和一个使能端,
    可以自动生成互补信号并插入内建死区时间,且内置下拉电阻,
    防止误触发。
    \item 采用UCC27211和BSC070N10NS
    \\ 
    UCC27211是自举升压型栅极驱动器,有两个驱动信号的输入端,
    可以独立控制上下MOS管,需要MCU生成互补信号和插入死区时间,
    控制程序更为复杂。
\end{enumerate}


综合以上两种方案,我们选择方案1。

\subsubsection{主控 MCU 的选择}
\begin{enumerate}
    \item 梁山派GD32F470ZGT6 
    \\
    梁山派采用基于 ARM Cortex-M4 内核的GD32F470ZGT6,
    主频可达240MHz,能够满足电源环路的高速运算,板上具有丰富外设,
    并引出多个GPIO接口,支持多种通信接口和外设,
    具有2.6Mbps的ADC,每个ADC具有19个通道,可供高速取样模拟信号,
    且开发板上具有大功率3.3V的输出,可以满足外设的供电需求,
    是国产芯片,封装的函数库简洁高效。
    \item STM32G431RBT6 
    \\ 
    基于 ARM Cortex-M系列 内核,
    主频可达 170MHz,具有52个GPIO接口,多个定时器,
    支持较多通信接口和外设,ADC DAC精度很高,
    但其使用的HAL库代码较为抽象难懂。    
    \item STC89C52 
    \\ 
    价格便宜,但基于 8051 内核,主频只有 12MHz,
    内存容量小,IO和定时器数量少。
\end{enumerate}


综合以上两种方案,我们选择方案2。

\subsubsection{电流采样方案的选择}
\begin{enumerate}
    \item INA199电流采样芯片 
    \\ 
    电路简单,成本低,但是需要采用电流取样电阻,
    电阻功耗大,而且承受的共模电压低,不能用高端电流取样,
    且电阻的阻值会随着温度的变化而变化,影响采样精度。
    \item 使用CC6903霍尔传感器 
    \\ 
    电路简单,成本低,芯片价格便宜,而且霍尔传感器共模电压高,
    且不需要采样电阻,损失能量少,不受温度影响,采样精度高。
    \item 使用HLW8032电能采样芯片  
    \\
    采样简单,不需要进行滤波和运算,直接输出有效值和功率值;
    缺点是采样精度低,数据刷新速度慢。
\end{enumerate}



综合以上两种方案,我们选择方案2。

\subsection{系统设计结构原理}
根据题目要求和设计方案,
本文设计了如图\ref{fig2}所示的系统结构原理图。
\begin{figure}[htbp]
\centering
\includegraphics[width=0.9\textwidth]{src/fig.png}
\caption{系统结构原理图}
\label{fig2}
\end{figure}


\section{系统理论分析与计算}
\subsection{逆变器电路原理与SPWM}
逆变是将直流电能转换为交流电能的过程,
逆变器是实现逆变的电路。本文采用H桥SPWM方式实现逆变,
SPWM是一种常用的逆变控制方法,
其基本思想是将正弦波信号与三角波信号进行比较,
当正弦波大于三角波时,输出高电平;当正弦波小于三角波时,
输出低电平。这样就可以得到近似于正弦波的波形。
\\
根据 SPWM 法的原理,可以推得
\begin{equation}
T_{on}=\frac{ARR-CCR}{ARR}=\frac{U_c\sin(\omega t_o)}{U}
\end{equation}
故单片机中寄存器的值应赋值为
\begin{equation}
CCR=ARR-\frac{U_c\sin (\omega t_o)}{U}\times ARR
\end{equation}

\subsection{逆变器并网运行原理与控制}
逆变器并网运行需要满足以下条件:
并网逆变器的输出电压幅值与电网电压幅值相等,频率与电网频率相等,
相位与电网电压相位相等并网逆变器具有良好的并网保护功能,
本文采用相位移位控制方式实现逆变器并网运行,
本设计通过过零比较器检测电网零点,
并将信号送入单片机进行计数,得到电网周期和频率。
通过霍尔传感器采样电流,并将信号送入单片机进行处理,
单片机将电网与逆变器差异对比计算出误差信号,
并通过PID算法进行反馈调节。根据反馈调节结果,
调节SPWM信号的相位差,实现逆变器与电网之间的并网。


\section{电路与程序设计}
\subsection{电路设计}

\subsubsection{电路总体设计}
电路总体框图如图\ref{fig4}所示,主控通过电流采样,以及过零比较器
调整相位,控制逆变器的输出,使其满足并网要求。
\begin{figure}[H]
\centering
\includegraphics[width=0.7\textwidth]{src/fig4.png}
\caption{电路总体框图}
\label{fig4}
\end{figure}

\subsubsection{交流电压采样}
如图\ref{fig5}所示,交流电压取样采用把交流电压先经过电阻分压,
然后使用LM358作为电压跟随器,
然后经过SGM8632运放进行反向和抬升,
使原本的交流电压落在$0\sim 3.3V$的范围之内,
经过RC滤波电路后输入到ADC进行采样。
\begin{figure}[H]
\centering
\includegraphics[width=0.8\textwidth]{src/fig5.png}
\caption{交流电压采样设计图}
\label{fig5}
\end{figure}

\subsubsection{过零比较器}
如图\ref{fig8}所示,采用SGM8632运放把输入的交流电经过电阻分压之后,
过零比较器输出上升沿,单片机进入中断,
调整$\sin$函数的计数值为0,即一个周期的开始,
使程序中的正弦信号与外部电网同步。

\begin{figure}[htbp]
    \centering
    \begin{minipage}{0.45\textwidth}
        \centering
        \includegraphics[width=1\textwidth]{src/fig8.png}
        \caption{过零比较器设计图}
        \label{fig8}
    \end{minipage}
    \quad
    \begin{minipage}{0.45\textwidth}
        \centering
        \includegraphics[width=1\textwidth]{src/fig11.png}
        \caption{电流取样设计图}
        \label{fig11}
    \end{minipage}
\end{figure}
\subsubsection{电流取样}
如图\ref{fig11}所示,电流取样采用CC6903霍尔传感器。

\subsubsection{H桥}
H桥电路独立设计,板上集成栅极驱动电路和母线电容,
预留控制信号的输入端口和电源的输入输出端口。
\begin{figure}[htbp]
    \centering
    \begin{minipage}{0.45\textwidth}
        \centering
        \includegraphics[width=1\textwidth]{src/fig6.png}
        \caption{H桥设计图}
    \end{minipage}
    \quad
    \begin{minipage}{0.45\textwidth}
        \centering
        \includegraphics[width=1\textwidth]{src/fig7.png}
        \caption{栅极驱动电路图}
    \end{minipage}
\end{figure}
\newpage
\subsubsection{辅助电源}
辅助电源采用XL8015和LM2596开关电源芯片,
XL8015支持最高$80V$的输入,输出功率最大$12V$,
作为第一级电源把主电源的电压降到12V,
用于给MOS的栅极驱动芯片和串口屏供电,
然后LM2596把$12V$的电压降为$5V$,用于给主控板供电,
主控板上有TMI3411把$5V$降到$3.3V$,
用于给GD32F470ZGT6供电和运放供电,
使用SGM3204电荷泵芯片产生$-5V$的电源,给运放供电。
详见附录图\ref{fig10},图\ref{fig9}。

\subsubsection{SPWM的生成}
根据系统功能需求,本文SPWM生成流程如图\ref{fig12}所示。
代码详见附录\ref{set_spwm}。
\begin{figure}[H]
    \centering
    \includegraphics[width=0.8\textwidth]{src/fig12.png}
    \caption{SPWM生成程序框图}
    \label{fig12}
\end{figure}

\subsubsection{电压、电流控制环路}
电压、电流控制环路如图\ref{fig3}所示,当定时器溢出后,
会进行ADC采样,采样完成后,会进行增量式PID运算,
从而更新PWM值,实现电压、电流的控制。
代码详见附录\ref{pid},\ref{current}。
\begin{figure}[H]
    \centering
    \includegraphics[width=0.8\textwidth]{src/fig3.png}
    \caption{电压、电流控制环路}
    \label{fig3}
\end{figure}

\section{系统测试及结果分析}

\subsection{测试方案}
为了验证系统的性能是否满足设计要求,本设计采用了以下测试方案,
测试仪器见表\ref{tab1}。
\begin{enumerate}
    \item 断开S2,闭合S1,
    测试逆变器1向负载供电时的$U_o\ I_o\ f\ THD$以及负载调整率等。
    \item 断开S2闭合S1,
    测试两个逆变器并联向负载供电时的$U_o\   I_o\   f$等。
    \item 断开S2闭合S1,
    测试两个逆变器并联且并网后的输出电流、输出电流比值、
    其误差绝对值。
\end{enumerate}

\begin{table}[H]
    \centering
    \begin{minipage}{0.45\textwidth}
    \caption{测试仪器}
    \begin{tabular}{cccc}
    \toprule
    序号 & 仪器名称 & 型号 & 数量  \\ \midrule
    1 & 示波器 & DS2202A & 1\\
    2 & 可调电源 & IT6722A & 2 \\
    3 & 功率分析仪 & TA333H & 1 \\
    4 & 万用表 &  ZT102A & 1
    \\ \bottomrule
    \label{tab1}
    \end{tabular}
    \end{minipage}
    \quad
    \begin{minipage}{0.45\textwidth}
    \centering
    \caption{测试结果}
    \begin{tabular}{@{}cccc@{}}
    \toprule
    序号 & 测试指标 & 测试结果 & 结论 \\ \midrule
    \multirow{4}{*}{1} & $U_o$ & $24.1V$ & \multirow{4}{*}{满足要求} \\
     & $I_o$ & $2.1A$ &  \\
     & $f_o$ & $50.1Hz$ &  \\
     & $THD$ & $1.8\%$ &  \\ \midrule
    \multirow{3}{*}{2} & $U_o$ & $24.0V$ & \multirow{3}{*}{满足要求} \\
        & $I_o$ & $4.1A$ &  \\
        & $f_o$ & $50.1Hz$ &  \\ \midrule
    \multirow{3}{*}{3} & $I_{o1}$ & $2.1A$ & \multirow{3}{*}{满足要求} \\
        & $I_{o2}$ & $1.0A$ &  \\
        & $K$ & $2.1$ &  \\ \bottomrule
    \end{tabular}
    \label{tab2}
    \end{minipage}
\end{table}
\newpage
\subsection{测试结果及分析}
系统的性能经过测试,具有较高的效率和稳定性,
能够实现逆变器的并联运行和并网运行,
测试结果见表\ref{tab2}及图\ref{res}。

\section{结论}
本设计的主要创新点和优点如下使用高性能的单片机 GD32F470 控制逆变器,
充分利用其高速运算和多功能外设的特点,
实现多任务的并行处理和高效率的数据传输。
使用主从控制方式实现逆变器并联运行,简化了调节机制,
提高了同步精度和灵活性。使用相位移位控制方式实现逆变器并网运行,
简化了控制方法,提高了响应速度和可靠性。
使用触摸串口屏作为人机交互界面,方便了用户对系统的操作和监测。

\nocite{*}
\printbibliography


\newpage

\appendix
\section{附录}
\subsection{主要元器件清单}
\noindent 梁山派GD32F470开发板 \quad 淘晶驰X2系列串口屏 
\quad EG2104S \quad SGM3204 \quad SGM8632 
\quad BSC070N10NS\quad LM358\quad 74HC245
\quad XL8015降压模块 \quad LM2596降压模块
\quad 磁环电感\quad CBB22电容
\quad 可调电阻 \quad 稳压模块 \quad 双路继电器模块 
\quad CC6903\quad 电阻 \quad 电感 
\subsection{部分程序代码如下}
\subsubsection{增量式PID运算}\label{pid}
\begin{verbatim}
p->Ek_0 = p->Ref - p->Fdb;  //计算误差
//误差小于设定死区,不进行PID计算,保持上一次输出
if( fabs(p->Ek_0) < p->Ek_Dead )
{
    p->Increm = 0;
}
else
{
    p->Increm = (   p->a0 * p->Ek_0  \
                - p->a1 * p->Ek_1 \
                + p->a2 * p->Ek_2 ); //PID增量计算  
}
p->Output += p->Increm; //计算输出
p->Ek_2        = p->Ek_1; //保存k-2误差
p->Ek_1        = p->Ek_0; //保存k-1误差
if(p->Output > p->OutMax)
{
    p->Output   =  p->OutMax; //最大限幅
    return;
}
if(p->Output < p->OutMin)
{
    p->Output   =  p->OutMin; //最小限幅
    return;
}
\end{verbatim} 
\subsubsection{PWM的设置}\label{set_spwm}
\begin{verbatim}
void set_spwm(float pwm)//通道2对应输出为正极输出
{
    if(pwm>0)
    {
        tim_ch_output_pulse_config(TIMER1,TIMER_CH_2,(uint16_t)pwm);
        tim_ch_output_pulse_config(TIMER1,TIMER_CH_3,0);
    }
    else
    {
        pwm=-pwm;
        tim_ch_output_pulse_config(TIMER1,TIMER_CH_2,0);
        tim_ch_output_pulse_config(TIMER1,TIMER_CH_3,(uint16_t)pwm);   
    }
}
\end{verbatim}
\subsubsection{电流环运算}\label{current}
\begin{verbatim}
void power_ctrl_ACcurrent()
{
    gPID_CurrentOutLoop.Ref=i_acout*_SQRT2*SIN_TIME;
    gPID_CurrentOutLoop.Fdb=i_in1.Value;
    float Bias,kp=200,ki=250;  //定义相关变量
    static float pwm_out, Last_bias; //静态变量

    Bias=gPID_CurrentOutLoop.Ref-gPID_CurrentOutLoop.Fdb; //求速度偏差

    pwm_out+=kp*(Bias-Last_bias)+ki*Bias;  //增量式PI控制器
    Last_bias=Bias;  
        pwm_out+=SIN_TIME*DP_PWM_PER*0.1f;
    if(pwm_out>0.90f * DP_PWM_PER)
    {
        pwm_out=0.90f * DP_PWM_PER;
    }
    if(pwm_out<-0.90f * DP_PWM_PER)
    {
        pwm_out=-0.90f * DP_PWM_PER;
    }
    set_spwm(pwm_out);
    if(cnt_spwm==100) ip_inverter=i_in1.Value/_SQRT2;
}
\end{verbatim}


\subsection{阻性负载测试结果}\label{res}
\begin{figure}[htbp]
    \centering
    \begin{minipage}{0.45\textwidth}
        \centering
        \includegraphics[width=0.9\textwidth]{src/result1.jpg}
        \caption{功率分析仪测试结果}
    \end{minipage}
    \qquad
    \begin{minipage}{0.45\textwidth}
        \centering
        \includegraphics[width=0.9\textwidth]{src/result2.jpg}
        \caption{示波器波形}
    \end{minipage}
    \\
    \includegraphics[width=0.3\textwidth]{src/circuit.jpg}
    \caption{部分电路连接图}
\end{figure}
\newpage
\subsection{$3.3V$辅助电源设计图}
\begin{figure}[h]
    \centering
    \includegraphics[width=0.7\textwidth]{src/fig10.png}
    \caption{3.3V辅助电源}
    \label{fig10}
\end{figure}
\subsection{$\pm 5V$电源设计图}
\begin{figure}[h]
    \centering
    \includegraphics[width=0.7\textwidth]{src/fig9.png}
    \caption{$\pm 5V$电源}
    \label{fig9}
\end{figure}
% 结束文档
\end{document}
